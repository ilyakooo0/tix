\documentclass[12pt]{article}

\usepackage{xcolor}
\usepackage[hidelinks]{hyperref}

\newcommand{\todo}{\addcontentsline{toc}{subsection}{{\color{red}\textbf{TODO}}}{\LARGE\color{red}\textbf{TODO}}}

\newcounter{req}

\newcommand{\req}[1][]{\refstepcounter{req}\label{req:#1}\paragraph{\fbox{\textsf{\textsc{REQ}-\thereq}}}}
\newcommand{\refreq}[1]{\hyperref[req:#1]{\fbox{\textsf{REQ-\ref*{req:#1}}}}}

\title{Technical Specification for the Static Analyzer for the Nix Expression Language}
\author{Kostyuchenko Ilya}
\date{}

\begin{document}

\maketitle

\newpage

\tableofcontents

\newpage

\section{Purpose}

The purpose of the \emph{Static analyzer for the Nix Expression Language} (later referred to as \emph{Tix} for brevity) is to simplify development in the Nix expression language.

\section{Scope}

The project consists of developing the following components:

\paragraph{The Tix type checker} is a program which is designed to statically analyze the given Nix expression language source code and deduce the types of subexpressions used within while detecting some errors made by the programmer (type mismatch errors, which usually arise due to programming mistakes).

\section{Product perspective}

\subsection{System interfaces}

The application runs on the latest versions of macOS and Linux.

\subsection{User interfaces}

The application does not provide a user interface.

\subsection{Hardware interfaces}

The application does not provide a hardware interface.

\subsection{Software interfaces}

The application allows interactions through standard UNIX input-output capabilities.

\subsection{Communications interfaces}

The application communicates with the user by reading input from the file system or terminal emulator, and outputting results to the standard UNIX output handle.

\subsection{Memory}

The memory required to run the system depends on the size of the project being processed. Typical projects should are processed within a 2 GB memory limit.

\subsection{Operations}

The application supports the following operations:

\begin{enumerate}
  \item Typecheck the current file and report errors
\end{enumerate}

\subsection{Site adaptation requirements}

No site adaptations are required.

\subsection{Interfaces with services}

No interaction with services in performed.

\section{Product functions}

The application provides the following functionality to the user:

\begin{enumerate}
  \item Checks user-supplied programs for type incompatibilities
  \item Reports the resulting types of user-supplied expressions
\end{enumerate}

\section{User characteristics}

The user of the application will need intermediate technical skills to set up the program. More precisely, the user will need to have basic understanding of how to use their operating system and command line.

\section{Limitations}

\subsection{Interface to other applications}

Other applications can be interfaced with through standard input-output handles.

\subsection{Criticality of the application}

The software should not be used for critical applications as it is not capable of eliminating all runtime errors, or even eliminating all runtime type errors due to the dynamic nature of the underlying Nix expression language\footnote{Type systems inherently restrict some valid programs, and some expressions might not be well-typed, but execute without errors.}.

\section{Assumptions and dependencies}

The contents of this document assume the following:

\begin{enumerate}
  \item The application will be used on a UNIX-like system that is supported by the Glasgow Haskell Compiler.
  \item All dependencies of the files being processed should be locally downloaded by Nix.
\end{enumerate}

\section{Apportioning of requirements}

The application is a self-contained executable file that, when executed, functions as a typechecker.

\section{Specified requirements}

\req[restrictons] The system should perform all static program analysis (instances of such analysis link to this requirement) under the following conditions:
\begin{enumerate}
  \item it is possible to perform analysis without (partially) evaluating expressions
  \item all input data linked to from the expressions being analyzed is locally downloaded by Nix
\end{enumerate}

\req[typechecking] The system should perform type checking for expression where under conditions from \refreq{restrictons}.

\req The system should analyze the types and presence of attributes in attribute sets under conditions from \refreq{restrictons}.

\req The system should report encountered typechecking errors.

\req The system should report the use of undefined terms through means defined by the Language Server Protocol under conditions from \refreq{restrictons}.

\section{Usability requirements}

\req The system should be able to process expressions from the nixpkgs repository.

\req The system should never fail in unrecoverable ways in external imported expressions – if there are type errors in external code the system should resolve the errors automatically.

\section{Performance requirements}

\req The system should process incoming requests within 1 minute.

\section{External interfaces}

The only external interfaces of the system are standard input-output handles.

\paragraph{The purpose} of this interface is to provide easy way to provide programs to typecheck as inputs.

\section{Functions}

The fundamental functions of the system are queries: get the type of an expression.

\subsection{Generating the type of an expression}

\begin{enumerate}
  \item the system queries the expression at the requested position in the source code from the \emph{analysis context}
  \item the system queries the type of the expression from the \emph{analysis context}
  \item the system returns the type
\end{enumerate}

\subsection{Generating the definition of a variable}

\begin{enumerate}
  \item the system queries the expression at the requested position in the source code from the \emph{analysis context}
  \item the system asserts that the expression is a variable reference
  \item the system queries the definition of the given variable reference from the \emph{analysis context}
  \item the system returns the definition
\end{enumerate}

\subsection{Generating completions}

\begin{enumerate}
  \item the system queries the variables in scope at the requested position in the source code from the \emph{analysis context}
  \item the system formulates a list of possible completions
  \item the system filters the list of possible completions based on the prefix already typed by the user
  \item the system returns the list of filtered possible completions
\end{enumerate}

\subsection{Generating the analysis context}

\begin{enumerate}
  \item the current file is parsed
  \item dependencies are gathered from the current file and the analysis context is recursively generated for all dependencies
  \item the current file is typechecked
        \begin{enumerate}
          \item type constraints are recursively generated for all subexpressions
          \item encountered variable definitions are gathered
          \item encountered variable references are gathered
          \item undefined variable definitions are gathered
          \item type constraints are generated for the given top-level expression
          \item type constraints are sequentially solved. If a typing error is encountered, the error is gathered with other errors and the problematic type is replaced with a polymorphic type to allow further typechecking
        \end{enumerate}
  \item all errors gathered during typechecking are reported to the user through means defined by the Language Server Protocol
  \item all analysis contexts are merged into a single analysis context
  \item the resulting analysis context is the returned
\end{enumerate}

\subsection{Error handling}

If the system encounters a typing error in the user's code, then the error is returned to the user.

If the system encounters an internal (as opposed to a user error) the error is written to the standard error output.

\section{Design constraints}

The main design constraint is the fact that the Nix expression language is a dynamic language without an underlying static type system. This implies that there will be correct programs that will be ill-typed. The system should be designed in such a way as to gracefully handle such situations.

\section{Software system attributes}

\subsection{Reliability}

\begin{enumerate}
  \item The system should automatically recover from any user error after the user resolves the error in code.
  \item The system should not enter unrecoverable states.
\end{enumerate}

\subsection{Maintainability}

The system should utilize free monad algebraic effects for managing side-effects as they improve maintainability by decoupling implementation from abstraction.

\subsection{Portability}

\begin{enumerate}
  \item The system should not contain host-dependant code
  \item The system should be written in the Haskell programming language which is not host-dependent
  \item The system should be compiled with the Glasgow Haskell Compiler as it supports a wide range of platforms
\end{enumerate}

\section{Verification}

The following approaches and methods are planned to be employed to qualify the software:

\begin{enumerate}
  \item Automated hand-written unit-tests
  \item Manual testing
\end{enumerate}

\section{Supporting information}

\subsection{Nix}

Nix is a powerful package manager for Linux and other Unix systems that makes package management reliable and reproducible.\footnote{Taken from \url{https://nixos.org}}.

\subsection{Nix expression language}

The Nix expression language is a purely functional dynamic strictly typed programming language used for configuration in the Nix package manager. It does not support defining custom types – all types are defined within the interpreter. The main purpose of the language is to produce \emph{derivation} – a description of how to build an artifact\footnote{\url{https://nixos.org/manual/nix/stable/\#ssec-derivation}}.

\subsection{The problem to be solved}

When working with the Nix expression language it is often the case that imported expressions (nixpkgs) are quite large, and it is not obvious what needs to be called with what arguments. It is often necessary to inspect the source code of imported files to grok how it should be used and if the current use is correct.

This project aims to lighten the burden of development in the Nix expression language by providing some level of machine-generated introspection into the imported expressions and provide some level of verification of the correctness of the implementation.

\end{document}
